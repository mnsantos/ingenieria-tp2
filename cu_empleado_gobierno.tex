\subsection{Empleado de gobierno}

~

\cu{Ingresando nueva estación}{Empleado de gobierno}{}{12}

\begin{center}
    \centering
    \begin{tabular}{ | p{11cm} | p{6cm} | }
    	\multicolumn{1}{c}{\cellcolor{black!30}\textbf{Curso normal}} & 
    	\multicolumn{1}{c}{\cellcolor{black!30}\textbf{Curso alternativo}} \\ \hline
    	1- El empleado de gobierno indica al sistema que desea agregar una nueva estación & \\ \hline
    	2- El sistema despliega una interfaz con campos a completar necesarios para el registro de la 
    	nueva estación: nombre, dirección y capacidad de la misma. & \\ \hline
    	3- El empleado completa los datos y los envía al sistema & 3.1- Alguno de los campos es incorrecto:
    	Ya existe una estación en el sistema con dicho nombre o dirección, o la capacidad es incorrecta:
    	menor a la mínima - superior a la máxima. Volver al paso 3. \\ \hline
    	4- El sistema le indica al empleado que la estación fue registrada correctamente
		\underline{Postcondición :} Estación agregada & \\ \hline
    \end{tabular}
\end{center}

~

\cu{Consultando estadísticas}{Empleado de gobierno}{}{13}

\begin{center}
    \centering
    \begin{tabular}{ | p{11cm} | p{6cm} | }
    	\multicolumn{1}{c}{\cellcolor{black!30}\textbf{Curso normal}} & 
    	\multicolumn{1}{c}{\cellcolor{black!30}\textbf{Curso alternativo}} \\ \hline
    	1- El empleado de gobierno indica al sistema que desea consultar una estadística. & \\ \hline
    	2- El sistema despliega una interfaz con las posibles estadísticas a consultar. & \\ \hline
    	3- El empleado selecciona la estadística de interés, junto con un rango temporal sobre el cual
    	desea consultar. (Último día, semana, mes, año o histórico) & \\ \hline
    	4- El sistema despliega la información solicitada. & \\ \hline
		\underline{Postcondición :} Estadística consultada & \\ \hline
    \end{tabular}
\end{center}

~

\cu{Consultando sugerencias}{Empleado de gobierno}{}{14}

\begin{center}
    \centering
    \begin{tabular}{ | p{11cm} | p{6cm} | }
    	\multicolumn{1}{c}{\cellcolor{black!30}\textbf{Curso normal}} & 
    	\multicolumn{1}{c}{\cellcolor{black!30}\textbf{Curso alternativo}} \\ \hline
    	1- El empleado de gobierno indica al sistema que desea ver sugerencias hechas por los usuarios. & \\ \hline
    	2- El sistema despliega una interfaz con varias opciones: Por un lado puede elegir consultar. 
    	las sugerencias de todos los usuarios, o seleccionar uno específico ingresando su DNI. Al mismo
    	tiempo puede filtrar las consultas por rango temporal (Realizadas el último día, semana, mes, año
    	o histórico). & \\ \hline
    	3- El empleado selecciona el modo de consulta. & 3.1- El usuario ingresa un DNI que no pertenece a ningún usuario. Volver al paso 2. \\ \hline
    	4- El sistema envía las sugerencias solicitadas. & \\ \hline
	\underline{Postcondición :} Sugerencias consultadas & \\ \hline
    \end{tabular}
\end{center}

~

\cu{Actualizando estado de la empresa de transportes}{Empleado de gobierno}{}{15}

\begin{center}
    \centering
    \begin{tabular}{ | p{11cm} | p{6cm} | }
    	\multicolumn{1}{c}{\cellcolor{black!30}\textbf{Curso normal}} & 
    	\multicolumn{1}{c}{\cellcolor{black!30}\textbf{Curso alternativo}} \\ \hline
    	1- El empleado de gobierno indica al sistema que desea ingresar información relevante respecto a la empresa de transportes & \\ \hline
    	2- El sistema despliega una interfaz con dos posibles opciones para completar: Advertir inactividad de la empresa durante cierto rango de tiempo,
    	o actualizar la cantidad de camiones disponibles para traslados durante cierto rango de tiempo & \\ \hline
    	3- El empleado de gobierno selecciona la opción que desea e ingresa los datos. & \\ \hline
	\underline{Postcondición :} Actualización completada & \\ \hline
    \end{tabular}
\end{center}

Si el empleado de gobierno ingresara Inactividad de la empresa entre 15:00 y 17:00hs estaría indicándole al sistema que la empresa de transportes no realizará ningún
traslado de bicicletas durante dicho horario.
Si el empleado ingresara 20 camiones entre las 15:00 y 17:00hs estaría indicándole al sistema que la empresa de transportes destinará 20 de sus camiones para traslados
de bicicletas durante dicho horario.

~

\cu{Ingresando nuevas bicicletas}{Empleado de gobierno}{}{16}

\begin{center}
    \centering
    \begin{tabular}{ | p{11cm} | p{6cm} | }
    	\multicolumn{1}{c}{\cellcolor{black!30}\textbf{Curso normal}} & 
    	\multicolumn{1}{c}{\cellcolor{black!30}\textbf{Curso alternativo}} \\ \hline
    	1- El empleado de gobierno indica al sistema que desea registrar el ingreso de nuevas bicicletas & \\ \hline
    	2- El sistema despliega una interfaz en donde permite seleccionar la cantidad de bicicletas, el tipo de las mismas y el rango Id correspondiente.& \\ \hline
    	3- El empleado de gobierno ingresa los datos & 3.1-  Alguno de los campos es incorrecto. Por ejemplo:
        ya existe una bicicleta en el sistema con el mismo ID. Volver al paso 3. \\ \hline
    	4- El sistema informa que los datos han sido correctamente actualizados & \\ \hline
    	\underline{Postcondición :} Nuevas bicicletas ingresadas & \\ \hline
    \end{tabular}
\end{center}

~

\cu{Validando registro}{Empleado de gobierno}{}{17}

\begin{center}
    \centering
    \begin{tabular}{ | p{11cm} | p{6cm} | }
    	\multicolumn{1}{c}{\cellcolor{black!30}\textbf{Curso normal}} & 
    	\multicolumn{1}{c}{\cellcolor{black!30}\textbf{Curso alternativo}} \\ \hline
    	1- El empleado de gobierno indica al sistema que desea obtener el listado de las personas registradas no verificadas & \\ \hline
    	2- El sistema envía el listado & \\ \hline
    	3- El empleado selecciona alguna de las entradas para validarla & \\ \hline
    	4- El sistema despliega la información presentada por el usuario al momento del registro & \\ \hline
    	5- El empleado indica que la información es válida & 5.1- El empleado encuentra una anomalía en los datos e indica qué información es inválida. El sistema notifica por mail al usuario
    	informándole los datos de registro inválidos y le solicita que se registre nuevamente \\ \hline
    	6- El sistema informa al empleado de gobierno que la validación se completó exitosamente & \\ \hline
    	\underline{Postcondición :} Registación de usuario completa & \\ \hline
    \end{tabular}
\end{center}

Luego de que un usuario se registra los datos quedan sujetos a verificación por parte de un empleado de gobierno. El mismo chequea consistencia entre la documentación presentada.
En caso de encontrar alguna anomalía invalida al registro y el sistema envía una notificación al usuario. En el caso de que la validación sea exitosa, el sistema le envía un mail
al usuario informándole que la validación ha sido completada con éxito.

~

\cu{Ingresando nuevo empleado}{Empleado de gobierno}{}{18}

\begin{center}
    \centering
    \begin{tabular}{ | p{11cm} | p{6cm} | }
        \multicolumn{1}{c}{\cellcolor{black!30}\textbf{Curso normal}} & 
        \multicolumn{1}{c}{\cellcolor{black!30}\textbf{Curso alternativo}} \\ \hline
        1- El empleado de gobierno indica al sistema que desea ingresar un nuevo empleado & \\ \hline
        2- El sistema despliega una interfaz en donde permite seleccionar la estación en donde desea incorporarlo & \\ \hline
        3- El empleado selecciona alguna de las estaciones & \\ \hline
        4- El sistema despliega una interfaz para ingresar los datos del empleado (Nombre, apellido, DNI, rango horario) & \\ \hline
        5- El empleado completa el formulario & 5.1- El sistema detecta errores en la información suministrada. Por ejemplo: empleado ya registrado en alguna estación. Volver al paso 5. \\ \hline
        6- El sistema informa al empleado de gobierno que el ingreso se completó exitosamente & \\ \hline
        \underline{Postcondición :} Ingreso de empleado completo & \\ \hline
    \end{tabular}
\end{center}

~

\cu{Dando de baja empleado}{Empleado de gobierno}{}{19}

\begin{center}
    \centering
    \begin{tabular}{ | p{11cm} | p{6cm} | }
        \multicolumn{1}{c}{\cellcolor{black!30}\textbf{Curso normal}} & 
        \multicolumn{1}{c}{\cellcolor{black!30}\textbf{Curso alternativo}} \\ \hline
        1- El empleado de gobierno indica al sistema que desea dar de baja a un empleado & \\ \hline
        2- El sistema despliega una interfaz en donde permite seleccionar la estación de la cual se desea remover al empleado& \\ \hline
        3- El empleado selecciona alguna de las estaciones & \\ \hline
        4- El sistema despliega una lista de los empleados de dicha estación & \\ \hline
        5- El empleado selecciona a uno de los empleados que muestra la interfaz \\ \hline
        6- El sistema informa al empleado de gobierno que se removió al empleado exitosamente & \\ \hline
        \underline{Postcondición :} Baja de empleado completa & \\ \hline
    \end{tabular}
\end{center}

~

\cu{Dando de baja estación}{Empleado de gobierno}{}{20}

\begin{center}
    \centering
    \begin{tabular}{ | p{11cm} | p{6cm} | }
        \multicolumn{1}{c}{\cellcolor{black!30}\textbf{Curso normal}} & 
        \multicolumn{1}{c}{\cellcolor{black!30}\textbf{Curso alternativo}} \\ \hline
        1- El empleado de gobierno indica al sistema que desea dar de baja una estación & \\ \hline
        2- El sistema despliega una interfaz en donde permite seleccionar la estación a remover& \\ \hline
        3- El empleado selecciona alguna de las estaciones & \\ \hline
        4- El sistema informa al empleado de gobierno que se removió la estación exitosamente & \\ \hline
        \underline{Postcondición :} Baja de estación completa & \\ \hline
    \end{tabular}
\end{center}




