\subsection{Empleado de gobierno}

~

\cu{Ingresando nueva estación}{Empleado de gobierno}{}

\begin{center}
    \centering
    \begin{tabular}{ | p{11cm} | p{6cm} | }
    	\multicolumn{1}{c}{\cellcolor{black!30}\textbf{Curso normal}} & 
    	\multicolumn{1}{c}{\cellcolor{black!30}\textbf{Curso alternativo}} \\ \hline
    	1- El empleado de gobierno indica al sistema que desea agregar una nueva estación & \\ \hline
    	2- El sistema despliega una interfaz con campos a completar necesarios para el registro de la 
    	nueva estación: nombre, dirección y capacidad de la misma. & \\ \hline
    	3- El empleado completa los datos y los envía al sistema & 3.1- Alguno de los campos es incorrecto:
    	Ya existe una estación en el sistema con dicho nombre o dirección, o la capacidad es incorrecta:
    	menor a la mínima - superior a la máxima. Volver al paso 3. \\ \hline
    	4- El sistema le indica al empleado que la estación fue registrada correctamente
		\underline{Postcondición :} Estación agregada & \\ \hline
    \end{tabular}
\end{center}

~

\cu{Consultando estadísticas}{Empleado de gobierno}{}

\begin{center}
    \centering
    \begin{tabular}{ | p{11cm} | p{6cm} | }
    	\multicolumn{1}{c}{\cellcolor{black!30}\textbf{Curso normal}} & 
    	\multicolumn{1}{c}{\cellcolor{black!30}\textbf{Curso alternativo}} \\ \hline
    	1- El empleado de gobierno indica al sistema que desea consultar una estadística. & \\ \hline
    	2- El sistema despliega una interfaz con las posibles estadísticas a consultar. & \\ \hline
    	3- El empleado selecciona la estadística de interés, junto con un rango temporal sobre el cual
    	desea consultar. (Último día, semana, mes, año o histórico) & \\ \hline
    	4- El sistema despliega la información solicitada. & \\ \hline
		\underline{Postcondición :} Estadística consultada & \\ \hline
    \end{tabular}
\end{center}

~

\cu{Consultando sugerencias}{Empleado de gobierno}{}

\begin{center}
    \centering
    \begin{tabular}{ | p{11cm} | p{6cm} | }
    	\multicolumn{1}{c}{\cellcolor{black!30}\textbf{Curso normal}} & 
    	\multicolumn{1}{c}{\cellcolor{black!30}\textbf{Curso alternativo}} \\ \hline
    	1- El empleado de gobierno indica al sistema que desea ver sugerencias hechas por los usuarios. & \\ \hline
    	2- El sistema despliega una interfaz con varias opciones: Por un lado puede elegir consultar. 
    	las sugerencias de todos los usuarios, o seleccionar uno específico ingresando su DNI. Al mismo
    	tiempo puede filtrar las consultas por rango temporal (Realizadas el último día, semana, mes, año
    	o histórico). & \\ \hline
    	3- El empleado selecciona el modo de consulta. & 3.1- El usuario ingresa un DNI que no pertenece a ningún usuario. Volver al paso 2. \\ \hline
    	4- El sistema envía las sugerencias solicitadas. & \\ \hline
		\underline{Postcondición :} Sugerencias consultadas & \\ \hline
    \end{tabular}
\end{center}


