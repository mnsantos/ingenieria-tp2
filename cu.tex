\subsection{Usuario}

~

\cu{Iniciando trámite de registración}{Usuario}{}{1}

\begin{center}
    \centering
    \begin{tabular}{ | p{11cm} | p{6cm} | }
    	\multicolumn{1}{c}{\cellcolor{black!30}\textbf{Curso normal}} & 
    	\multicolumn{1}{c}{\cellcolor{black!30}\textbf{Curso alternativo}} \\
		\hline
		1- El sistema despliega una interfaz gráfica con campos a completar
		por el usuario: Nombre completo, Número de DNI, Domicilio, Mail y un teléfono
		para contacto en caso de ser necesario &  \\ \hline
		2 - El usuario ingresa los datos en el sistema & 2.1 - El usuario ya se encuentra ingresado en el sistema
		o hay una inconsistencia con los datos de otro usuario almacenado en la base de datos. Fin de C.U \\ \hline
		3 - El sistema registra al usuario en el sistema, almacena los datos en la base de datos y le anuncia al usuario que la registración se completará cuando los datos se validen y que se le comunicará de dicha operación via email.& \\ \hline
		\underline{Postcondición :} Trámite de registración iniciado & \\ \hline
    \end{tabular}
\end{center}

Una vez que el registro es realizado, los datos quedan sujetos a verificación por parte de un empleado de gobierno.

~

\cu{Logueándose}{Usuario}{}{18}
\begin{center}
    \centering
    \begin{tabular}{ | p{11cm} | p{6cm} | }
    	\multicolumn{1}{c}{\cellcolor{black!30}\textbf{Curso normal}} & 
    	\multicolumn{1}{c}{\cellcolor{black!30}\textbf{Curso alternativo}} \\
		\hline
		1- El sistema despliega una interfaz gráfica con campos a completar
		por el usuario: Nombre completo, Número de DNI, Domicilio, Mail y un teléfono
		para contacto en caso de ser necesario &  \\ \hline
		2 - El usuario ingresa los datos en el sistema & 2.1 - El usuario ya se encuentra ingresado en el sistema
		o hay una inconsistencia con los datos de otro usuario almacenado en la base de datos. Fin de C.U \\ \hline
		3 - El sistema registra al usuario en el sistema y almacena los datos en la base de datos & \\ \hline
		\underline{Postcondición :} Usuario registrado & \\ \hline
    \end{tabular}
\end{center}

~

\cu{Consultando stock/tiempo}{Usuario}{Trámite de registración iniciado}{2}
\begin{center}
    \centering
    \begin{tabular}{ | p{11cm} | p{6cm} | }
    	\multicolumn{1}{c}{\cellcolor{black!30}\textbf{Curso normal}} & 
    	\multicolumn{1}{c}{\cellcolor{black!30}\textbf{Curso alternativo}} \\
		\hline
		1- El usuario le informa al sistema que desea realizar una consulta acerca de la disponibilidad
		de bicicletas en una determinada estación & \\ \hline
		2- El sistema despliega una lista con las estaciones que es posible consultar & 
		2.1- El usuario no encuentra en la lista la estación que está buscando porque la misma no se encuentra en
		funcionamiento, con lo cual o bien procede a consultar disponibilidad en una estación cercana (ir al paso 3), o
		decide finalizar la consulta (Fin de C.U)\\ \hline
		3- El usuario selecciona la estación sobre la cual desea consultar. & \\ \hline
		4- El sistema envía información acerca de la disponibilidad en la estación seleccionada al momento de la consulta, utilizando distintos colores para reforzar el contenido del mensaje. En el caso de haber muchas bicicletas disponibles utiliza el color verde. Si hay una cantidad moderada de bicicletas utiliza el color naranja. Finalmente,
		si la cantidad de bicicletas disponibles está por agotarse utiliza el color rojo. Si por el contrario no hay
		bicicletas disponibles en la estación el sistema informa este hecho al usuario y envía un tiempo estimado en el cual
		es probable que el stock vuelva a ser positivo. & \\ \hline
		\underline{Postcondición :} Consulta realizada & \\ \hline
    \end{tabular}
\end{center}	

~

\cu{Consultando propio estado}{Usuario}{Usuario logueado}{3}
\begin{center}
    \centering
    \begin{tabular}{ | p{11cm} | p{6cm} | }
    	\multicolumn{1}{c}{\cellcolor{black!30}\textbf{Curso normal}} & 
    	\multicolumn{1}{c}{\cellcolor{black!30}\textbf{Curso alternativo}} \\
		\hline
		1- El usuario le informa al sistema que desea consultar su estado & \\ \hline
		2- El sistema le despliega una interfaz con varias opciones a consultar:
		Si el usuario está penalizado actualmente, el historial de 
		penalizaciones y el historial de alquileres. & \\ \hline
		3- El usuario indica la opción que desea consultar & \\ \hline
		4- El sistema le envía al usuario la información solicitada.
		\underline{Postcondición :} Consulta individual realizada & \\ \hline
    \end{tabular}
\end{center}

~

\cu{Agregando sugerencia}{Usuario}{Usuario logueado}{4}
\begin{center}
    \centering
    \begin{tabular}{ | p{11cm} | p{6cm} | }
    	\multicolumn{1}{c}{\cellcolor{black!30}\textbf{Curso normal}} & 
    	\multicolumn{1}{c}{\cellcolor{black!30}\textbf{Curso alternativo}} \\
		\hline
		1- El usuario le informa al sistema que desea enviar una sugerencia & \\ \hline
		2- El sistema despliega un cuadro de texto en donde el usuario puede escribir los comentarios,
		mostrando también las pautas del correcto uso de dicha herramienta (Solicita
		utilizar el espacio con
		responsabilidad evitando el uso indebido del lenguaje).
		También muestra ejemplos de comentarios que han ayudado a la mejora de las ciclovías. & \\ \hline
		3- El usuario ingresa sus comentarios y los envía al sistema & \\ \hline
		4- El sistema le indica al usuario que las sugerencias han sido almacenadas & \\ \hline
		\underline{Postcondición :} Sugerencia enviada & \\ \hline
    \end{tabular}
\end{center}




