Luego de llegar a un acuerdo preliminar respecto a los requerimientos con los cuales debe cumplir nuestro sistema decidimos
realizar una nueva iteración en donde tratamos de abordar aspectos un poco más específicos relacionados con la funcionalidad de 
nuestro software.

Para ello realizamos cuatro diagramas complementarios, aprovechando la semántica y sintaxis particular de cada uno de ellos 
para poder expresar distintos aspectos de nuestra máquina:

Por un lado, utilizamos un modelo conceptual que nos permitió identificar los conceptos principales de la red de ciclovías
en general y determinar la forma en que se relacionan entre ellos.

Al mismo tiempo generamos un diagrama de casos de uso, en el cual detallamos las distintos tipos de interacciones con el
sistema que tendrán los actores. Por cada caso de uso agregamos una descripción en donde expresamos con palabras la forma
en que se llevará a cabo dicha interacción.

En relación con este último punto agregamos un diagrama de actividad, con el objetivo de mostrar el vínculo entre los
participantes de la red de ciclovías. Este análisis se realiza en un contexto temporal, mediante una especie de flujo
en el que van ocurriendo los distintos eventos. Cada uno de los participantes juega un rol particular y sus acciones
son agrupadas en distintos andariveles.

Finalmente utlizamos el modelo de Finite State Machines para representar el proceso de alquiler de bicicletas junto con 
otros muy relacionados, como por ejemplo la reposición de bicicletas.

