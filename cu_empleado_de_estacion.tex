\cu{Consultando datos de usuario}{Empleado de estación}{}

\begin{center}
    \centering
    \begin{tabular}{ | p{11cm} | p{6cm} | }
    	\multicolumn{1}{c}{\cellcolor{black!30}\textbf{Curso normal}} & 
    	\multicolumn{1}{c}{\cellcolor{black!30}\textbf{Curso alternativo}} \\
		\hline
		1- El empleado de estación indica al sistema que desea ver los datos de un usuario. &  \\ \hline
		2- El sistema despliega una interfaz gráfica con un campo DNI a completar correspondiente al usuario a penalizar. &  \\ \hline
		3- El empleado de estación ingresa el DNI del usuario. &  
		3.1- El sistema indica que el usuario no existe y permite volver al paso anterior para ingresar el DNI. Volver al paso 3. \\ \hline
		4- El sistema muestra los datos del usuario correspondientes al DNI ingresado: Nombre, e-mail, teléfono y dirección. También muestra el estado del alquiler actual o un formulario para ingresar un nuevo alquiler(en caso de que no haya ningún alquiler activo) y un formulario para ingresar penalizaciones. &
		4.1- Si el usuario no es correcto, el sistema permite reingresar el DNI. Volver al paso 3. \\ \hline		
		\underline{Postcondición :} Datos de usuario consultados & \\ \hline
    \end{tabular}
\end{center}

~

\cu{Registrando retiro de bicicleta}{Empleado de estación}{}
\begin{center}
    \centering
    \begin{tabular}{ | p{11cm} | p{6cm} | }
    	\multicolumn{1}{c}{\cellcolor{black!30}\textbf{Curso normal}} & 
    	\multicolumn{1}{c}{\cellcolor{black!30}\textbf{Curso alternativo}} \\
		\hline
		1- El empleado de estación indica al sistema que desea consultar los datos de un usuario. Usa Consultando datos de usuario. &  \\ \hline
		2- El empleado de estación completa el formulario de alquiler, ingresando el ID de la bicicleta. Luego envía el formulario al sistema. & \\ \hline
		\underline{Postcondición :} Retiro de bicicleta registrado & \\ \hline
    \end{tabular}
\end{center}

~

\cu{Ingresando penalización}{Empleado de estación}{}

\begin{center}
    \centering
    \begin{tabular}{ | p{11cm} | p{6cm} | }
    	\multicolumn{1}{c}{\cellcolor{black!30}\textbf{Curso normal}} & 
    	\multicolumn{1}{c}{\cellcolor{black!30}\textbf{Curso alternativo}} \\
		\hline
		1- El empleado de estación indica al sistema que desea consultar los datos de un usuario. Usa consultando datos de usuario. &  \\ \hline
		2. El empleado de estación completa el formulario de penalizaciones, ingresando el tipo de penalización y algún comentario pertinente a la penalización. Luego envía el formulario al sistema. & \\ \hline
		\underline{Postcondición :} Usuario penalizado & \\ \hline
    \end{tabular}
\end{center}


~

\cu{Modificando stock de estación}{Empleado de estación}{}
\begin{center}
    \centering
    \begin{tabular}{ | p{11cm} | p{6cm} | }
    	\multicolumn{1}{c}{\cellcolor{black!30}\textbf{Curso normal}} & 
    	\multicolumn{1}{c}{\cellcolor{black!30}\textbf{Curso alternativo}} \\
		\hline
		1- El empleado de estación indica al sistema que desea modificar el stock de estación. & \\ \hline
		2- El sistema despliega un formulario para ingresar por separado los listados de IDs de las bicicletas que se van a agregar y remover de la estación. & \\ \hline
		3- El empleado de estación ingresa en el formulario los IDs de las bicicletas que se agregan o remueven de la estación en los campos correspondientes y envía el formulario al sistema. & \\ \hline
		\underline{Postcondición :} Stock de estación modificado & \\ \hline
    \end{tabular}
\end{center}	

~

\cu{Ingresando bicicleta en mal estado}{Empleado de estación}{}
\begin{center}
    \centering
    \begin{tabular}{ | p{11cm} | p{6cm} | }
    	\multicolumn{1}{c}{\cellcolor{black!30}\textbf{Curso normal}} & 
    	\multicolumn{1}{c}{\cellcolor{black!30}\textbf{Curso alternativo}} \\
		\hline
		1- El empleado de estación indica al sistema que desea modificar el estado de una bicicleta. & \\ \hline
		2- El sistema muestra un formulario con los campos ID y estado, donde ID es el de la bicicleta y estado es un campo de texto para ingresar el detalle del estado de la bicicleta. & \\ \hline
		3- El empleado de estación completa el formulario y lo envía. & \\ \hline
		4- El sistema luego calcula y muestra en que fecha y horario se repondrá la bicicleta. & \\ \hline
		\underline{Postcondición :} Bicicleta en mal estado ingresada & \\ \hline
    \end{tabular}
\end{center}

