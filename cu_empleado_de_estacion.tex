\subsection{Empleado de estación}

Para las acciones en donde interactúan el empleado de estación y el usuario (principalmente a la hora de retirar una bicicleta
y luego devolverla)
es estrictamente necesario que se haya completado
el proceso de registración para este último (es decir, que se hayan sometido los datos a un proceso de validación por parte
del empleado de gobierno). Esto es consistente respecto a nuestros requerimientos ya que contribuye a la seguridad
del sistema de la red de ciclovías.

~

\cu{Validando autenticación}{Empleado de estación}{}
\begin{center}
    \centering
    \begin{tabular}{ | p{11cm} | p{6cm} | }
    	\multicolumn{1}{c}{\cellcolor{black!30}\textbf{Curso normal}} & 
    	\multicolumn{1}{c}{\cellcolor{black!30}\textbf{Curso alternativo}} \\
		\hline
		1- El empleado de estación indica al sistema que desea validar los datos del usuario. &  \\ \hline
		2- El sistema despliega una interfaz gráfica con un campo DNI a completar correspondiente al usuario. &  \\ \hline
		3- El empleado de estación ingresa el DNI del usuario. &  
		3.1- El sistema indica que el usuario se encuentra parcialmente registrado (finaliza CU) o que no existe y permite volver al paso anterior para ingresar el DNI. Volver al paso 3. \\ \hline
		4- El sistema indica que el usuario es válido. & \\ \hline
		\underline{Postcondición :} Autenticación de usuario validada & \\ \hline
    \end{tabular}
\end{center}

~

\cu{Consultando datos de usuario}{Empleado de estación}{Autenticación de usuario validada}
\begin{center}
    \centering
    \begin{tabular}{ | p{11cm} | p{6cm} | }
    	\multicolumn{1}{c}{\cellcolor{black!30}\textbf{Curso normal}} & 
    	\multicolumn{1}{c}{\cellcolor{black!30}\textbf{Curso alternativo}} \\
		\hline
		1- El empleado de estación indica al sistema que desea ver los datos del usuario. &  \\ \hline
		2- El sistema despliega una interfaz gráfica con un campo DNI a completar correspondiente al usuario. &  \\ \hline
		3- El empleado de estación ingresa el DNI del usuario. & \\ \hline  
		3.1- El sistema indica que el usuario no existe y permite volver al paso anterior para reingresar el DNI. Volver al paso 3. \\ \hline
		4- El sistema muestra los datos del usuario correspondientes al DNI ingresado: Nombre, e-mail, teléfono y dirección. También muestra el estado del alquiler actual o un formulario para ingresar un nuevo alquiler(en caso de que no haya ningún alquiler activo), si está penalizado actualmente y un formulario para ingresar penalizaciones. & \\ \hline		
		\underline{Postcondición :} Datos de usuario consultados & \\ \hline
    \end{tabular}
\end{center}

~

\cu{Registrando retiro de bicicleta}{Empleado de estación}{Autenticación de usuario validada}
\begin{center}
    \centering
    \begin{tabular}{ | p{11cm} | p{6cm} | }
    	\multicolumn{1}{c}{\cellcolor{black!30}\textbf{Curso normal}} & 
    	\multicolumn{1}{c}{\cellcolor{black!30}\textbf{Curso alternativo}} \\
		\hline
		1- El empleado de estación consulta los datos del usuario. Usa CU: Consultando datos de usuario. &  \\ \hline
		2- El empleado de estación completa el formulario de alquiler, ingresando el ID de la bicicleta. Luego envía el formulario al sistema. & \\ \hline
		3- El sistema indica que el alquiler fue registrado correctamente. & \\ \hline
		\underline{Postcondición :} Retiro de bicicleta registrado & \\ \hline
    \end{tabular}
\end{center}

~

\cu{Ingresando penalización}{Empleado de estación}{Autenticación de usuario validada}
\begin{center}
    \centering
    \begin{tabular}{ | p{11cm} | p{6cm} | }
    	\multicolumn{1}{c}{\cellcolor{black!30}\textbf{Curso normal}} & 
    	\multicolumn{1}{c}{\cellcolor{black!30}\textbf{Curso alternativo}} \\
		\hline
		1- El empleado de estación consulta los datos del usuario. Usa CU: Consultando datos de usuario. &  \\ \hline
		2. El empleado de estación completa el formulario de penalizaciones, ingresando el tipo de penalización y algún comentario pertinente a la penalización. Luego envía el formulario al sistema. & \\ \hline
		3- El sistema indica que la penalización fue ingresada correctamente. & \\ \hline
		\underline{Postcondición :} Usuario penalizado & \\ \hline
    \end{tabular}
\end{center}

El tiempo que el usuario dure penalizado es calculado por el sistema a partir del tipo de penalización indicado.

~

\cu{Modificando stock de estación}{Empleado de estación}{}
\begin{center}
    \centering
    \begin{tabular}{ | p{11cm} | p{6cm} | }
    	\multicolumn{1}{c}{\cellcolor{black!30}\textbf{Curso normal}} & 
    	\multicolumn{1}{c}{\cellcolor{black!30}\textbf{Curso alternativo}} \\
		\hline
		1- El empleado de estación indica al sistema que desea modificar el stock de estación. & \\ \hline
		2- El sistema despliega un formulario para ingresar por separado los listados de IDs de las bicicletas que se van a agregar o remover de la estación. & \\ \hline
		3- El empleado de estación ingresa en el formulario los IDs de las bicicletas que se agregan o remueven de la estación en los campos correspondientes y envía el formulario al sistema, e indica que él es el responsable de la
		transacción en dicha estación & 
		3.1- El sistema detecta inconsistencia en el listado de Id's ingresado por lo que el empleado de la estación vuelve a ingresar los datos. Volver al paso 3.\\ \hline
		4- El sistema indica el stock fue modificado correctamente. & \\ \hline
		\underline{Postcondición :} Stock de estación modificado & \\ \hline
    \end{tabular}
\end{center}	

El empleado de estación modifica el stock cuando la empresa de transportes ingresa o retira bicicletas de su estación.
Cuando se movilizan bicicletas, se modifica el stock en 2 estaciones: origen y destino (ver diagrama de clases). Para ambas
debe haber un responsable de la transacción. I.E.: Si se movilizan 20 bicicletas de la estación A a la B, el empleado de 
la estación A que modifica el stock debería hacerse responsable de corroborar que 20 y solo 20 bicicletas
(cuyos id's coincidan con los del pedido) sean retiradas, y el empleado de la estación B que reciba las bicicletas debería
corroborar que sean esas mismas 20 bicicletas las que llegan. Esto contribuye al correcto funcionamiento del sistema, y a la 
posibilidad de asignar responsabilidades a las distintas partes del proceso de traslado (en donde participan distintos agentes).

~

\cu{Ingresando bicicleta en mal estado}{Empleado de estación}{}
\begin{center}
    \centering
    \begin{tabular}{ | p{11cm} | p{6cm} | }
    	\multicolumn{1}{c}{\cellcolor{black!30}\textbf{Curso normal}} & 
    	\multicolumn{1}{c}{\cellcolor{black!30}\textbf{Curso alternativo}} \\
		\hline
		1- El empleado de estación indica al sistema que desea modificar el estado de una bicicleta. & \\ \hline
		2- El sistema muestra un formulario con los campos ID y estado, donde ID es el de la bicicleta y estado es un campo de texto para ingresar el detalle del estado de la bicicleta. & \\ \hline
		3- El empleado de estación completa el formulario y lo envía. &
		3.1- El sistema detecta inconsistencia en el Id ingresado por lo que el empleado de la estación vuelve a ingresar los datos. Volver al paso 3. \\ \hline
		4- El sistema indica que se ingresó correctamente el cambio de estado de la bicicleta. & \\ \hline
		\underline{Postcondición :} Bicicleta en mal estado ingresada & \\ \hline
    \end{tabular}
\end{center}

~

\cu{Registrando devolución de bicicleta}{Empleado de estación}{Autenticación de usuario validada}
\begin{center}
    \centering
    \begin{tabular}{ | p{11cm} | p{6cm} | }
    	\multicolumn{1}{c}{\cellcolor{black!30}\textbf{Curso normal}} & 
    	\multicolumn{1}{c}{\cellcolor{black!30}\textbf{Curso alternativo}} \\
		\hline
		1- El empleado de estación consulta los datos del usuario. Usa CU: Consultando datos de usuario. &  \\ \hline
		2- El empleado de estación actualiza el estado del alquiler del usuario para reflejar que la bicicleta fue devuelta exitosamente. &  \\ \hline
		3- Si la bicicleta está en mal estado, el empleado de estación penaliza al usuario(extiende CU: Ingresando Penalización) e ingresa la bicicleta en mal estado al sistema(extiende CU: Ingresando bicicleta en mal estado). &  \\ \hline	
		4- Si el tiempo de alquiler fue excedido, el empleado de estación penaliza al usuario por demora(extiende CU: Ingresando Penalización).
		\underline{Postcondición :} Devolución de bicicleta registrada & \\ \hline
    \end{tabular}
\end{center}

